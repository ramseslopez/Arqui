\documentclass[12pt, letterpaper]{article}
  \usepackage[utf8]{inputenc}
  \usepackage[left = 2.5cm, right = 2.5cm, top = 3cm, bottom = 3cm]{geometry}
  \usepackage[T1]{fontenc}
  \usepackage{listings}


  \author{Hernández Ferreiro Enrique Ehécatl \\
          López Soto Ramses Antonio}

        \title{Práctica 2: Introducción a C \\
                {\small Organización y Arquitectura de Computadoras}}

                \date{27 de febrero de 2019}

  \begin{document}
    \maketitle
    \section{Introducción}
    Se desarrolló un pequño programa en el lenguaje de programación en C para
    poder calcular las medidas de tendencia central de un conjunto de datos dados
    por el usuario.
    \section{Desarrollo}
    Se implementaron en total tres métodos con los cuales se calcula:
    \begin{center}
            \framebox{Medidas de tendencia central}
    \end{center}

    $$D=\{\delta_1, \delta_2, ..., \delta_n\}$$

    \begin{center}
            Media aritmética   \hspace{2cm}   $A(D)=\frac{1}{n}\displaystyle\sum_{i=1}^{n}\delta_{i}$
    \end{center}

    \begin{center}
            Media armónica   \hspace{2cm}   $H(D)=\frac{n}{\displaystyle\sum_{i=1}^{n}\frac{1}{\delta_{i}}}$
    \end{center}

    \begin{center}
            Media geométrica   \hspace{2cm}   $G(D)=\sqrt[n]{\displaystyle\prod_{i=0}^{n}\delta_{i}}$
    \end{center}

    \begin{center}
            $H(D) \leq G(D) \leq A(D)$
    \end{center}

    En código de C se implementó de la mejor siguiente forma:

    \lstinputlisting{practica2.c}
    \section{Conclusión}
    Al implementar estas funciones sencillas en C, adquirimos conocimientos básicos
    del lenguaje, en particular el uso de apuntadores.
  \end{document}
