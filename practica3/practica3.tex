\documentclass[12pt, letterpaper]{article}
  \usepackage[utf8]{inputenc}
  \usepackage[left = 2.5cm, right = 2.5cm, top = 3cm, bottom = 3cm]{geometry}
  \usepackage[T1]{fontenc}
  \usepackage{amsthm}
  \usepackage{amsfonts}
  \usepackage{amsmath}
  \usepackage{amssymb}
  \usepackage{graphicx}
  \usepackage{tikz,xcolor}
  \usepackage{karnaugh-map}


  \author{Hernández Ferreiro Enrique Ehécatl \\
          López Soto Ramses Antonio}

        \title{Práctica 3: Circuitos combinacionales \\
                {\small Organización y Arquitectura de Computadoras}}

                \date{24 de febrero de 2019}

        \begin{document}
        \maketitle
        \section{Introducción}

          \hspace{.5cm}
          A lo largo de los años nos hemos preguntado cómo es que las computadoras
          funcionan si no piensan, no razonan, sólo rlizan las series de instrucciones
          que nosotros les indicamos. Esto se responde con la lógica proposicional.\vspace{.3cm}

          Pero, ¿la lógica proposicional? En el año de 1854 un hombre llamado Geoge Boole
          desarrolló el \textit{álgebra booleana} que abrió camino para el desarrollo de la 
          \textit{lógica dígital}, con la cual nuestras computadoras funcionan.

        \section{Desarrollo}

          \hspace{.5cm}
          Para el desarrollo de los circuitos se utilizó el software \textit{Logisim}. La primer
          parte consistió en el desarrolo de la implicación lógica y la segunda en la construcción de un comparador
          de dos bits.

          \subsection{Implicación lógica}

            \hspace{.5cm}
            Para la construcción de la implicación lógica se ocuparon:

            \begin{itemize}
              \item 2 pines de entrada
              \item 1 pin de salida
              \item 3 transistores tipo PNP
              \item 3 transitores tipo NPN
            \end{itemize}

          \subsection{Comparador}

            \hspace{.5cm}
            Para la construcción del comparador se uso lo siguiente:
 
            \subsubsection{Fórmulas}
                \hspace{.5cm}
                El circuito funciona y se comporta como se muestra en la siguiente tabla, donde $A=A_0A_1$ y $B=B_0B_1$:
                  \begin{table}[htbp]
                    \centering
                    \begin{tabular}{|c|c|c|c|c|c|c|}
                    \hline
                      \textbf{$A_0$} & \textbf{$A_1$} & \textbf{$B_0$} & \textbf{$B_1$} & \textbf{$A = B$} & \textbf{$A > B$} & \textbf{$A < B$} \\ \hline
                      \textbf{0} & \textbf{0} & \textbf{0} & \textbf{0} & \textbf{1} & \textbf{0} & \textbf{0} \\ \hline
                      \textbf{0} & \textbf{0} & \textbf{0} & \textbf{1} & \textbf{0} & \textbf{0} & \textbf{1} \\ \hline
                      \textbf{0} & \textbf{0} & \textbf{1} & \textbf{0} & \textbf{0} & \textbf{0} & \textbf{1} \\ \hline
                      \textbf{0} & \textbf{0} & \textbf{1} & \textbf{1} & \textbf{0} & \textbf{0} &\textbf{1} \\ \hline
                      \textbf{0} & \textbf{1} & \textbf{0} & \textbf{0} & \textbf{0} & \textbf{1} & \textbf{0} \\ \hline
                      \textbf{0} & \textbf{1} & \textbf{0} & \textbf{1} & \textbf{1} & \textbf{0} & \textbf{0} \\ \hline
                      \textbf{0} & \textbf{1} & \textbf{1} & \textbf{0} & \textbf{0} & \textbf{0} & \textbf{1} \\ \hline
                      \textbf{0} & \textbf{1} & \textbf{1} & \textbf{1} & \textbf{0} & \textbf{0} & \textbf{1} \\ \hline
                      \textbf{1} & \textbf{0} & \textbf{0} & \textbf{0} & \textbf{0} & \textbf{1} & \textbf{0} \\ \hline
                      \textbf{1} & \textbf{0} & \textbf{0} & \textbf{1} & \textbf{0} & \textbf{1} & \textbf{0} \\ \hline
                      \textbf{1} & \textbf{0} & \textbf{1} & \textbf{0} & \textbf{1} & \textbf{0} & \textbf{0} \\ \hline
                      \textbf{1} & \textbf{0} & \textbf{1} & \textbf{1} & \textbf{0} & \textbf{0} & \textbf{1} \\ \hline
                      \textbf{1} & \textbf{1} & \textbf{0} & \textbf{0} & \textbf{0} & \textbf{1} & \textbf{0} \\ \hline
                      \textbf{1} & \textbf{1} & \textbf{0} & \textbf{1} & \textbf{0} & \textbf{1} & \textbf{0} \\ \hline
                      \textbf{1} & \textbf{1} & \textbf{1} & \textbf{0} & \textbf{0} & \textbf{1} & \textbf{0} \\ \hline
                      \textbf{1} & \textbf{1} & \textbf{1} & \textbf{1} & \textbf{1} & \textbf{0} & \textbf{0} \\ \hline
                    \end{tabular}
                    \caption{Tabla de verdad del circuito}
                  \end{table}\vspace{.1cm}

                Usando la tabla de valores de verdad para los valores del bit A y B, sabemos
                que los números serán: (00 = 0, 01 = 1, 10 = 2, 11 = 3) entonces podemos
                establecer una relación entre A y B, en las celdas donde se cumpla la relación
                tendremos un 1 y donde no un 0, entonces procederemos a formar una expresión
                booleana.

                Las espresiones booleanas en cada uno de los casos son:
                \begin{itemize}
                  \item $(A=B)=\overline{A_0}\overline{A_1}\overline{B_0}\overline{B_1}+
                          \overline{A_0}A_1\overline{B_0}\overline{B_1}+A_0\overline{A_1}B_0\overline{B_1}
                          +\overline{A_0}\overline{A_1}\overline{B_0}\overline{B_1}$
                  \item $(A>B)=\overline{A_0}A_1\overline{B_0}\overline{B_1}+A_0\overline{A_1}\overline{B_0}\overline{B_1}
                          +A_0\overline{A_1}\overline{B_0}B_1+A_0A_1\overline{B_0}\overline{B_1}+
                          A_0A_1B_0\overline{B_1}$
                  \item $(A<B)=\overline{A_0}\overline{A_1}\overline{B_0}B_1+\overline{A_0}\overline{A_1}B_0\overline{B_1}
                          +\overline{A_0}\overline{A_1}B_0B_1+\overline{A_0}A_1B_0\overline{B_1}+
                          \overline{A_0}A_1B_0B_1+A_0\overline{A_1}B_0B_1$ 
                \end{itemize}

              \subsubsection{Mapas de Karnaugh}

                  Los mapas de karnaugh para minimizar las fórmulas son: \vspace{.6cm}

                  \begin{itemize}
                    \item $A=B$ \\
                      \begin{karnaugh-map}*[4][4][1][$B_0B_1$][$A_0A_1$]
                          \maxterms{1,2,3,4,6,7,8,9,11,12,13,14}
                          \minterms{0,5,10,15}
                          \autoterms[X]
                          \implicant{0}{0}
                          \implicant{5}{5}
                          \implicant{10}{10}
                          \implicant{15}{15}
                      \end{karnaugh-map} \vspace{.3cm}

                      Por lo que $(A=B)=\overline{A_0}\overline{A_1}\overline{B_0}\overline{B_1}+\overline{A_0}A_1\overline{B_0}B_1
                      +A_0\overline{A_1}B_0\overline{B_1}+A_0A_1B_0B_1$

                    \item $A>B$ \\
                      \begin{karnaugh-map}*[4][4][1][$B_0B_1$][$A_0A_1$]
                          \maxterms{0,1,2,3,5,6,7,10,11,15}
                          \minterms{4,8,9,14,12,13}
                          \autoterms[X]
                          \implicant{4}{12}
                          \implicant{12}{9}
                          \implicantedge{12}{12}{14}{14}
                      \end{karnaugh-map} \vspace{.1cm}

                      Por lo que $(A>B)=A_1\overline{B_0}\overline{B_1}+A_0A_1\overline{B_1}$\vspace{.3cm}

                    \item $A<B$\\
                    \begin{karnaugh-map}*[4][4][1][$B_0B_1$][$A_0A_1$]
                      \maxterms{0,4,12,8,5,13,9,15,14,10}
                      \minterms{1,2,3,7,6,11}
                      \autoterms[X]
                      \implicant{1}{3}
                      \implicant{3}{6}
                      \implicantedge{3}{3}{11}{11}
                    \end{karnaugh-map} \vspace{.1cm}

                    Por lo que $(A<B)=\overline{A_0}\overline{A_1}B_1+\overline{A_0}B_0+\overline{A_1}B_0B_1$\vspace{.3cm}
                        
                  \end{itemize}
          
        \section{Conclusiones}
            
            \hspace{.5cm}
            La construcción de circuitos dígitales es una tarea importante para quienes se dedican
            a la fabricación de computadoras. A pesar que nosotros nos enfocamos más en el cómo nunca 
            es malo saber un poco de cómo funcionan los componentes de nuestras computadoras.

            En resumen, la práctica mos ayudó a entender un poco mejor el comprtamiento de algunos 
            componentes de nuestras herramientas de trabajo.

            \subsection{Preguntas}

            \begin{itemize}
              \item[1.] ¿Cuál es el procedimiento a seguir para desarrollar 
                        un circuito que resuelva un problema que involucre 
                        lógica combinacional?\vspace{.1cm}

                        \textbf{R.} Hacer una tabla de verdad con todos lo 
                        casos posibles, extraer aquellas combinaciones que 
                        den el resultado deseado, juntar las expresiones y 
                        reducir usando mapas de Karnaugh.\vspace{.1cm} 

              \item[2.] Si una función de conmutación se evalúa a más ceros 
                        que unos ¿es conveniente usar mintérminos o maxtérminos? 
                        ¿En el caso que se evalué a más unos que ceros?\vspace{.1cm}

                        \textbf{R.} Más ceros usamos mintérminos y para más unos 
                        usamos maxtérminos.\vspace{.1cm}

              \item[3.] Analizando el trabajo realizado, ¿cuáles son los inconvenientes 
                        de desarrollar los circuitos de forma manual?\vspace{.1cm}

                        \textbf{R.} Que muchas veces abstraes un funcionamiento mentalmente 
                        pero resulta en fallas que no consideraste además de ser más complicado 
                        y se causan cortos más fácilmente.\vspace{.1cm} 
            \end{itemize}



        \end{document}
