\documentclass[12pt, letterpaper]{article}
  \usepackage[utf8]{inputenc}
  \usepackage[left = 2.5cm, right = 2.5cm, top = 3cm, bottom = 3cm]{geometry}
  \usepackage[T1]{fontenc}
  \usepackage{graphicx}
  \graphicspath{{images/}}
  \usepackage{listings}

  \author{Hernández Ferreiro Enrique Ehécatl \\
          López Soto Ramses Antonio}

        \title{Práctica 7: Convenciones de llamada a subrutina \\
                {\small Organización y Arquitectura de Computadoras}}
                \date{14 de abril de 2019}

  \begin{document}
  \maketitle
  \section{Desarrollo}
  El ejercicio 1 se muestra a continuación:\vspace{.1cm}

  \lstinputlisting{practica7.asm}\vspace{.3cm}

  El ejercicio 2 se muestra a continuación:\vspace{.1cm}

  \lstinputlisting{combinatoria.asm}\vspace{.3cm}

  \section{Conclusión}
  \subsection{Preguntas}
  \begin{itemize}
      \item[1.] ¿Qué utilidad tiene el registro \$fp? ¿Se puede prescidir de él? \\
                R. Sirve para preservar datos que no serán modificados por una llamada del sistema.
                Y sí, se puede prescindir de él, pues puede ser decrementado una vez que el procedimiento
                en el que se utilizó se acabe.
      \item[2.] Definimos como \textbf{subrutina nodo} a una subrutina que realiza
                una o más invocaciones a otras subrutinas y como \textbf{subrutina hoja}
                a una subrutina que no realiza llamadas a otras subrutinas.
                \begin{itemize}
                    \item[a)]   ¿Cuál es el tamaño mínimo que puede tener un marco para
                                una subrutina nodo? ¿Bajo qué condiciones ocurre?
                    \item[]     ¿Cuál es el tamaño mínimo que puede tener un marco para
                                una subrutina hoja? ¿Bajo qué condiciones ocurre?
                \end{itemize}
      \item[]
  \end{itemize}
  \end{document}
