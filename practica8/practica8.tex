\documentclass[12pt, letterpaper]{article}
  \usepackage[utf8]{inputenc}
  \usepackage[left = 2.5cm, right = 2.5cm, top = 3cm, bottom = 3cm]{geometry}
  \usepackage[T1]{fontenc}
  \usepackage{graphicx}
  \graphicspath{{images/}}
  \usepackage{listings}

  \author{Hernández Ferreiro Enrique Ehécatl \\
          López Soto Ramses Antonio}

        \title{Práctica 8: Excepciones \\
                {\small Organización y Arquitectura de Computadoras}}
                \date{9 de mayo de 2019}

  \begin{document}
    \maketitle
    \section{Introducción}
    Al ejecutar los programas que hacemos, es probable que se presenten casos en
    los cuales el programa falle y se rompa. Para evitar que esto ocurra utilizamos
    las llamadas \textit{excepiones}. \vspace{.1cm}

    Las \textit{excepiones}, por ejemplo en \textit{Java}, son aquellas que nos
    permiten que algún método le indique al código acerca de algún error que se
    pueda ocasionar durante la ejecución del programa. Cuando un excepción es
    lanzada la ejecución normal del programa y éste se altera.\vspace{.1cm}

    En MIPS MARS sucede algo similar, son señales que son enviadas al procesador
    por el hardware o el software que sucede al ocurrir un error por lo que el
    flujo normal de las instrucciones se ve alterado. Las excepciones en MIPS MARS
    se dividen de la siguiente manera:\vspace{.1cm}

    \begin{item}
      \item \underline{Excepciones internas} \vspace{.1cm}

            Para comunicarse con el procesador pues los dispositivos de entrada
            y salida ocasionan una \textit{interrupción}.

      \item \underline{Excepciones externas} \vspace{.1cm}

            Cuando ocurre un error o el procesador envía un mensaje al
            dispositivo.

    \end{item}\vspace{.1cm}



    En la práctica se utilizarán las excepciones para evitar que ocurran errores
    inesperados.\vspace{.3cm}

    \section{Desarrollo}
    En esta práctica se desarrolló una calculadora con notación postfija, es decir,
    primero se encuentran los operandos y al final los operadores, lo implica que
    la precedencia de operadores permanece sin la necesidad de separar las
    operaciones haciendo uso de paréntesis.\vspace{.1cm}

    \begin{table}
      \centering
      \begin{tabular}{ll}
          Notación normal (infija) & Notación postfija \\ \hline
               4 * 9                    & 4 9 *             \\
            3 * 1 + 2                & 3 1 * 2 +         \\
      \end{tabular}
    \end{table}\vspace{.2cm}

    \subsection*{Código}
    El código que se usó para la implementación de la calculadora es el siguiente:\vspace{.2cm}

    \lstinputlisting{practica8.asm}\vspace{.3cm}

    \section{Conclusiones}
    En esta práctica aprendimos el uso y funcionamiento básicos en MIPS MARS y
    también nos divertimos un poco.

    \subsection*{Preguntas}

    \begin{item}
      \item 1. En un procesador, ¿qué es el modo supervisor? ¿Qué funciones
                tiene? ¿Cómo se implementa?

                R. El modo supervisor es el que está permitido la ejecución de
                  cualquier instrucción de máquina como: la autorización de
                  interrupciones, el acceso a los registros utilizados por el
                  hardware u operaciones de entrada y de salida.

      \item 2. ¿Cuál es la relación entre una llamada al sistema y una excepción?

                R. Ambas se ejecutan en el modo supervisor.

      \item 3. ¿Qué es un vector de interrupciones?

                R. Es un vector en el que se encuentran los valores que apuntan
                   a la dirección en memoria del gestor de una interrupción. Es decir,
                   gestiona correctamente las interrupciones que se solicitan al
                   microprocesador.
    \end{item}

  \end{document}
